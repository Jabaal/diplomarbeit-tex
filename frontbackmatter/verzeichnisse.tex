%%%%%%%%%%%%%%% INHALTSVERZEICHNIS
\pagenumbering{Roman}% Seitenzahlen mit großen römischen Buchstaben beginnen
\pdfbookmark[section]{\contentsname}{toc}% Fügt in der Navigation (und nur in der PDF-Datei!) den Eintrag Inhaltsverzeichnis hinzu
\tableofcontents% Inhaltsverzeichnis erstellen
\newpage% 
%%%%%%%%%%%%%%% ABKÜRZUNGSVERZEICHNIS
\printglossary[% Verzeichnis einfügen
	type=abbreviations,% Abkürzungen hier einfügen 
	title=Abk\"urzungen,% neu benennen, sonst wird es "Akronyme" heißen, was nicht immer richtig ist
	style=long3col-booktabs,% Lange Tabelle mit 3 spalten und header und midrule (booktabs environment)
	nogroupskip% Kein vertikaler Abstand zwischen Buchstaben für die alphabetische Sortierung
	]%
\newpage%
%%%%%%%%%%%%%%%% VERZEICHNIS DER FORMELZEICHEN UND SYMBOLE
\printglossary[% Erstellt das Verzeichnis der Formelzeichen und Symbole hier
	type=symbols,% Symbolverzeichnis einfügen
	title={Formelzeichen und Symbole},% Umbenennen des Verzeichnisses
	style=altlongragged4col-booktabs,% longtab mit 4 spalten und einem header
	nogroupskip% Kein vertikaler Abstand für die alphabetische Sortierung
	]%
\newpage%
%%%%%%%%%%%%%%% ABBILDUNGSVERZEICHNIS
\listoffigures% Erstellt das Abbildungsverzeichnis
\newpage%
%%%%%%%%%%%%%%% TABELLENVERZEICHNIS
\listoftables% Erstellt das Tabellenverzeichnis
\newpage%