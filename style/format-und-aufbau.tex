%%% ALLGEMEINE FORMATIERUNGEN
\documentclass[
	twoside=false, % Doppelseitig schreiben oder nicht. Wichtig für automatische Bindekorrektur (BCOR).
	paper=a4, % Papiergröße festlegen, hier A4
	DIV=13, % Mit Hilfe der Option DIV=Faktor wird festgelegt, in wie viele Streifen die Seite horizontal und vertikal bei der Satzspiegelkonstruktion eingeteilt wird. DIV=calc nimmt einem diese Arbeit ab (hängt ja auch von der Schriftart und -größe ab usw.)
	BCOR=10mm, % Mit Hilfe der Option BCOR=Korrektur geben Sie den absoluten Wert der Bindekorrektur an, also die Breite des Bereichs, der durch die Bindung von der Papierbreite verloren geht. Am besten in der Druckerei nachfragen! Man kann einen Wert von ca. 10mm für Arbeiten bis 100 Seiten annehmen.
	parskip=false, % Wie Absätze gekennzeichnet werden. Zitat: "parskip=false Absätze werden durch einen Einzug der ersten Zeilen von einem Geviert (1 em) gekennzeichnet. Der erste Absatz eines Abschnitts wird nicht eingezogen." Alles andere ist hier nicht sinnvoll, siehe scrguide.pdf und Leitfaden_fuer_wiss_Dokumente.pdf!
	fontsize=11pt, % Schriftgröße
	listof=totoc, % Abbildungsverz. usw. werden in das Inhaltsverzeichnis aufgenommen
	bibliography=totoc % % Literaturverzeichnis wird in das Inhaltsverzeichnis aufgenommen
	]{scrreprt}
%
%
%%% HEADER und FOOTER DESIGN
\usepackage[ % Header und Footer designen
	autooneside=true, % ???
	headsepline=true, % Trennungsstrich für den Header (außer bei neuen Kapiteln und bei den Indizes)
	footsepline=true, % Trennungsstrich für den Footer (außer bei neuen Kapiteln und bei den Indizes)
	plainfootsepline=true % Trennungsstrich für den Footer bei neuen Kapiteln und bei den Indizes
	]{scrlayer-scrpage}
%
%
%%% WAS steht WO im HEADER und FOOTER
\pagestyle{scrheadings} % scrheadings einstellen, damit man alles weitere im Header und Footer richtig definieren kann
\clearscrheadfoot % Header und footer clearen (sonst hat man die Seitenzahl eventuell zwei mal im Footer...)
\automark[section]{chapter} % Von den ersten beiden Gliederungsebenen jeweils die niedrigste im Header anzeigen
\chead{\headmark} % Zentrierte Überschrift im Header
 \ofoot[\pagemark]{\pagemark}  % Seitenzahl für einseitigen Druck im Footer auf der rechten Seite platzieren, für zweiseitigen Druck immer außen platzieren
\addtokomafont{pagefoot}{\small} % Fußnoten etwas kleiner als die normale Schriftgröße setzen - zur Sicherheit den Befehl wiederholen - falls sich die Standard-Einstellungen ändern (safety first!)
%
%
%%% SPRACHE
\usepackage{polyglossia} % statt Babel nutzt man für xelatex polyglossia!
\setmainlanguage[
	variant=german, %es geht auch austrian oder swiss
	spelling=new, %neue Rechtschreibung ein/aus
	latesthyphen=true, %ausschalten bei bugs - neueste hyphenation
	babelshorthands=false %Trennung wie beim grauenvollen Paket Babel ausschalten		
	]%
{german}%
%
%
%%% SCHRIFTART
\usepackage{fontspec} % "Normale" Schriftarten können verwendet werden (die, die auf dem PC selbst installiert sind)
\setmainfont{Arial} % Schriftart festlegen
%
%
%%% ZEILENABSTAND
\usepackage{setspace} % Zeilenabstand einstellen mit setspace.sty - so werden auch in Fußzeilen die alten Abstände beibehalten! Siehe Sündenkatalog
\onehalfspacing % 1,5 Zeilig gewählt
% oder, falls dieses Paket mal nicht funktioniert:
% \linespread{1.2} % für Werte schaut man hier: https://www.sharelatex.com/learn/Paragraph_formatting#Line_spacing